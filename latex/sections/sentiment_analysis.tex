\section{Sentiment analysis}
\subsection{Results}
After projecting the values of LIWK markers onto the individual words, manual analysis was carried out on the tweets. The aim was to study the content of tweets in which words with a higher value of a specific marker are present. For this reason, the tweets were divided between inside and outside China and separated into the three periods taken into account.
\subsubsection{Negative emotions}
Considering the negative emotions marker, the following results emerged.\\
For the first period, for tweets outside China, the word with the highest value is \textit{'flyer'}. When analysing the news, it is noted that when the pandemic broke out, flyers were distributed in California urging people not to go to Asian restaurants, blaming them for spreading the virus. However, looking at tweets from within China, we see words such as \textit{'unfriendly'}, \textit{'bat'}, \textit{'criticize'}, all of which are contained in tweets complaining about the behaviour of America, which spreads fake news and blames China for the pandemic.\\
Concerning the second and third periods, it can be seen that in the tweets from outside China, the words that contain a higher negative emotion value according to the projection are no longer related to China but mainly concern the extensive damage caused by the pandemic. The tweets from Chinese official accounts, on the other hand, continue to accuse the rest of the world, and in particular America, of unfairly blaming China for the spread of the pandemic with the aim of blocking its development.